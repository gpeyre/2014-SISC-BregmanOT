% !TEX root = ../KL-Projections.tex


%%%%%%%%%%%%%%%%%%%%%%%%%%%%%%%%%%%%%%%%%%%%%%%%%%%%%%%
%%%%%%%%%%%%%%%%%%%%%%%%%%%%%%%%%%%%%%%%%%%%%%%%%%%%%%%
%%%%%%%%%%%%%%%%%%%%%%%%%%%%%%%%%%%%%%%%%%%%%%%%%%%%%%%
\section{Iterative Bregman Projections and Dykstra Algorithm}


In this paper, we focus on regularized generalized OT problems that can be re-cast in the form
\eql{\label{proj-inter}
	\min_{\pi \in \Cc}  \KLdiv{\pi}{\xi}
}
where $\xi$ is a given point in $\RR_+^{N\times N}$, and $\Cc$ is an intersection of closed convex sets
\eq{
	\Cc = \bigcap_{\ell=1}^L \Cc_\ell
}
such that $\Cc$ has nonempty intersection with $\RR_+^{N\times N}$. 

In the following, we extend the indexing of the sets by $L$-periodicity, so that they satisfy
\eq{
	\foralls n \in \NN, \quad \Cc_{n+L} = \Cc_{n}.
}


%%%%%%%%%%%%%%%%%%%%%%%%%%%%%%%%%%%%%%%%%%%%%%%%%%%%%%%
\subsection{Iterative Bregman Projections}
\label{sec-iterative-bregman}

In the special case where the convex sets $\Cc_\ell$ are  affine subspaces (note that nonnegativity constraints are already in the definition of the entropy), it is possible to solve~\eqref{proj-inter} by simply using iterative KL projections. Starting from $\pi^{(0)} = \xi$, one computes
\eql{\label{eq-iter-bregmanproj}
	\foralls n>0, \quad
	\iter{\pi} \eqdef \KLproj_{\Cc_n}( \pi^{(n-1)} ). 
}
One can then show that $\iter{\pi}$ converges towards the unique solution of~\eqref{proj-inter},
\eq{  
	\iter{\pi} \to \KLproj_\Cc(\xi) \quad \mbox{as $n \to \infty$}.
}
see~\cite{bregman1967relaxation}. 



%%%%%%%%%%%%%%%%%%%%%%%%%%%%%%%%%%%%%%%%%%%%%%%%%%%%%%%
\subsection{Dykstra's Algorithm}

When the convex sets $\Cc_\ell$ are not affine subspaces, iterative Bregman projections do not converge in general to the KL projection on the intersection. In contrast, Dykstra's algorithm~\cite{Dykstra83}, extended to the KL setting, does converge to the projection, see~\cite{bauschke-lewis}.

Dykstra's algorithm starts by initializing  
\eq{
	\pi^{(0)} \eqdef \xi
	\qandq
	q^{(0)}=q^{(-1)}= \dots =q^{(-L+1)} \eqdef \ones.
}
One then iteratively defines
\eql{\label{eq-iter-dystra}
	\iter{\pi} \eqdef  \KLproj_{\Cc_n}(\pi^{(n-1)} \odot q^{(n-L)}),
	\qandq
	\iter{q} \eqdef q^{(n-L)} \odot \frac{ \pi^{(n-1)} }{ \iter{\pi} }.
}
Recall here that $\odot$ and $\frac{\cdot}{\cdot}$ denotes entry-wise operations, see~\eqref{eq-entrywise}.

 
Dykstra algorithm converges to the solution of~\eqref{proj-inter}
\eq{  
	\iter{\pi} \to \KLproj_\Cc(\xi) \quad \mbox{as $n\to \infty$},
}
see~\cite{bauschke-lewis}.


% %% REMOVED %%

%%%%%%%%%%%%%%%%%%%%%%%%%%%%%%%%%%%%%%%%%%%%%%%%%%%%%%%
\subsection{Lifting over Product Spaces}


The algorithms defined through iterations~\eqref{eq-iter-dystra} and~\eqref{eq-iter-bregmanproj} are not symmetric with respect to the indexing of the constraints $\{\Cc_n\}$, i.e. if we re-order the constraints, the trajectories $\{\pi_n\}_n$ differ.

A way to obtain symmetric iterations can be achieved by lifting the problem over the product space $\bpi = (\pi^\ell)_{\ell=1}^L \in (\Si_N)^L$ and re-cast~\eqref{proj-inter} as the following projection
\eql{\label{eq-projinter-lifted}
	\min \enscond{
		\KLdivL{\bpi}{\bar\bpi} = \sum_{\ell=1}^L \la_\ell \KLdiv{\pi^\ell}{\bar\pi^\ell}
	}{
		\bpi \in \bCc_1 \cap \bCc_2
	}
}
where we introduced a a set of weights $\la \in \RR_+^L$, normalized so that $\sum_\ell \la_\ell=1$.

It is easy to check that similarly to the KL divergence, $\KL_\la$ is a also a $f$-divergence as defined in~\cite{}.

The lifted constraint sets are defined as
\begin{align*}
	\bCc_1 &= \enscond{ 
		\bpi = (\pi^\ell)_{\ell=1}^L \in (\Si_N)^L
	}{
		\forall \ell, \ell', \; \pi^\ell = \pi^{\ell'}
	} \\
	\bCc_2 &= \enscond{ 
		\bpi = (\pi^\ell)_{\ell=1}^L \in (\Si_N)^L
	}{
		\forall \ell, \; \pi^\ell \in \Cc_\ell
	} .
\end{align*}

The projection on $\bCc_2$ is simple to compute since the constraint is separable
\eq{
	\KLprojL_{\bCc_2}(\bar\bpi)
	= \pa{ 
		\KLproj_{\Cc_\ell}(\bar\pi_\ell)
	}_{\ell=1}^L
}
The projection on the diagonal set $\bCc_1$ is computed as described in the following proposition. 

\begin{prop}
	The projection $(\pi^\ell)_{\ell=1}^L = \KLprojL_{\Cc_1}(\bar\bpi)$ satisfies 
	\eq{
		\foralls \ell=1,\ldots,L, \quad
			\pi^\ell = \prod_{r=1}^L (\pi^r)^{\la_r}
	}
	where product and exponentiation should be understood component wise. 
\end{prop}
\begin{proof}
	TODO.
\end{proof}

It is then possible to use iterations~\eqref{eq-iter-bregmanproj} (when the sets $\Cc_\ell$ are affine) or \eqref{eq-iter-dystra} (for the general case) to solve the projection~\eqref{eq-projinter-lifted}, which defines iterations that differs from the one obtained when applying these algorithm directly to~\eqref{proj-inter}.


